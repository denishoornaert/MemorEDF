\section{Discussion \textcolor{ForestGreen}{and Limitation}}
\label{sec:discussion}
%When comparing the throughput that is experienced by the cores, the normal route always provides a higher throughput in comparison to the the loop-back-based path.
%However, when only comparing those loop-back-based paths, we can see that the price to pay for having \schim is largely compensated by the memory isolation the latter provides.
%In general, redirecting the cores traffic toward the PL side with \schim is interesting in combination with techniques such as  \textit{address bleaching} and \textit{zero-copy recoloring}.
\color{red}
Even though the throughput offered by the \emph{normal route} is higher, the
authors argue that comparing the latter's raw performance against \schim is unfair.
Redirecting the CPU-originated memory traffic through the PL side has a cost.
However, this cost is mainly linked to the implementation and the platform capabilities, elements that can be improved by optimization as well as a selection of more aggressive platforms.
The important aspect brought by the proposed framework, \schim, is its capability to individually manipulate memory transactions, opening the door to the study of novel memory scheduling policies.

The PL-to-PS feedback is an interesting regulating mechanism. Moreover, it is entirely transparent to the hypervisor's inmates and uses the fastest route possible to communicate with the PS side. Nonetheless, this feedback mechanism is coarse. Inset 1 of Figure \ref{fig:schim_behaviour_tdma} highlights perfectly this problem. Even though all the queues have been assigned a threshold of 4, the latter is often exceeded. The worst-case being queue 3 exceeding the threshold by 2 on the right side of the plot.
The thresholds used for the FIQ regulation requires to be fine-tuned manually by the user. Future extensions of the \schim will explore the implementation of schedulers capable of dynamically adapting the threshold to maximize the performance and improve the core isolation.

\color{ForestGreen}

By design, the PLiM module proposed in this paper, \schim, centralizes the
memory traffic and its scheduling. This centralization works on the platform
considered as there exist one and only one path between the the LLC and the
DRAM controller. In systems where multiple paths between the processing units
and the memory controllers exist, centralized system such as \schim should be
prohibited. In such platforms, a distributed approach should be prioritize in
order to benefit from all the parallelization offered.

While the \schim module does not strictly require the PS side to use a hypervisor,
Jailhouse has been extensively used for the evalution as it provides convinient tools.
For instance, the coloring capability of Jailhouse has been both used to partition
the cache and to identify the ower of each transactions (as presented in Section \ref{}).
However, the cache could not be partitioned and the cores identification could be
carried on by extracing any other portion of the transactions address such as the
Most Significant Bits. In addition, the second level of address translation is
convinent to completely and transparently redirect the inmates through the PL side.
Finally, the cores throttling via the FIQ routine can also be implemented without
requirering an hypervisor.

The cost in bandwidth caused by the routing is important and a serious drawback
agianst the adoption of \schim, despite abling the shaping of the traffic to
enforce a given policy.
The authors imput this drop of bandwidth to the path between the PS side module
populating the path from the core cluster to the HPM ports rather than the
frequency of the PL side itself. In fact, if one assumes a frequency of 250MHz
and one burst beat of 16 bytes per clock cycle, the PL side can sustain a
throughput of approximately 3.7GBps. This is in line with the throughput reported
in \cite{uiuc-xilinx-port-study}. Moreover, the frequency comparison should involve the
frequency of the main bus rather than the cores as on the platform used, the former
is clocked at 400MHz whereas the former is clocked at 1.5GHz. On the other hand,
while \schim does introduce extra latency and reduces the bandwidth, the authors
think that most of the penalty is comming from the two levels of interconnects
and fifos located on the path from the core cluster to the HPM ports.

The PL-to-PS feedback is an interesting regulating mechanism. Moreover, provided
that the FIQ lines are not used by the inmates, the mechanism is entirely transparent
to the hypervisor's inmates and uses the fastest route possible to communicate
with the PS side. As far as the authors know, the Linux inmate used to run the
benchmarks and the inmates used to stress the bus do not use the FIQ lines.
Nonetheless, it must be acknowledged that requiring FIQ lines for the correct
operation of \schim can enter into conflict with software that would also
require them.
The PL-to-PS feedback mechanism is coarse. Inset 1 of Figure
\ref{fig:schim_behaviour_tdma} highlights perfectly this problem. Even though all
the queues have been assigned a threshold of 4, the latter is often exceeded.
The worst-case being queue 3 exceeding the threshold by 2 on the right side of the
plot. These excesses can be imputed to the reaction time of the FIQ routine. The
thresholds used for the FIQ regulation requires to be fine-tuned manually by the
user. Future extensions of the \schim will have explore the implementation of
schedulers capable of dynamically adapting the threshold to maximize the performance
and improve the core isolation.
\color{black}
