\newpage
\section{Reviews}

    \subsection{Shepherding Requirements}
        \subsubsection{Please remove the discussion of the TS scheduler since it has logical errors in its current status as highlighted in the paper and shown in evaluation}
            \label{subsubsec:ts_removal}
            Every mention of the TS scheduler have been removed.
            \begin{itemize}
                \item line 303-304, 312-313: mentions removed
                \item line 444-465: whole section 5.5.3 removed
                \item Table 1: TS entries removed from configuration port
                \item line 541: mentions removed
                \item Figure 4: TS bar cluster removed
                \item line 557-558, 563-564: mentions removed
                \item line 581-582: mentions removed
                \item Figure 5: TS inset removed
                \item line 601-602, 603, 604-605: mentions removed
                \item Figure 6: TS inset removed
                \item line 624-632: TS trace snapshot discussion removed
                \item line 651-652, 661-665: TS isolation discussion removed
                \item line 735: mentions removed
            \end{itemize}

        \subsubsection{Add a discussion about the limitations of the work as highlighted by the reviewers}
            Discussion regarding the limitation and the nuancing of our approach have been integrated in section 7, "Discussion".
            The section is organized as follow:
            \begin{itemize}
                \item paragraph 1: discuss the overall place of \schim in system as a centralized point
                \item paragraph 2: discuss the requirements of \schim with respect to the hypervisor
                \item paragraph 3: discuss the bandwidth cost to pay for routing the traffic through the PL and discuss the reason of this drop.
                \item paragraph 4: discuss the pl-to-ps feedback and its implementation using FIQ lines
            \end{itemize}

        \subsubsection{Explain the value the proposed scheduler-in-the-middle is adding to the isolation use-case. In particular, address the following concern: you experiment with enforced DRAM bank partitioning, which is known by existing literature to enable such isolation on its own. Ideally, a comparison with the system, where scheduler-in-the-middle was applies without bank partitioning and the system where only bank partitioning was applied without the scheduler-in-the-middle will help distilling the effectiveness of each one.}


    \subsection{Reviewer A}
        \subsubsection{There is a rather throrough experimental evaluation, but it sometimes raises more questions than it answers. This suggests the experiments are not yet fully mature.}
            

        \subsubsection{Responses are not intercepted by SchIM, preventing response time analysis of different schedulers.}
            Having a clearer view on the response time of different memory components at the transaction level is one of the many features that can be made available by the PLiM approach. While the authors are interested in exploring such module in the future, we do not consider this as a requirement for \schim.

        \subsubsection{The TS scheduler has logic errors. It is unclear why it is still included in the paper.}
            Every elements linked to the TS policy have been removed. See Section \ref{subsubsec:ts_removal} for more information.

    \subsection{Reviewer B}
        \subsubsection{Serialization through HPMs and the PLIM design with a single central memory scheduler severly limits the approach and constraints the kind of systems for which memory scheduling can be explored.}
            The authors acknowledge this as a current limitation of \schim and agree that it must be discussed.

        \subsubsection{Given that PLIM and PL-side memory arbitrarion existed before, the originality of the work is limited.}
            PLiM existed before but it did not actively manipulate the transactions, it just changed some data on the fly. Regarding the PL-side arbitration, as far as the authors know, there does not exist widely available PS-side plocy that can act on cores individually. This would be the case if the cores are not clusterd together behind a LLC and a bus master by them selves. With that regard, we think that \schim offers novelty.

        \subsubsection{It is not clear to me whether the evaluation correctly reflects the performance of the baseline.}


        \subsubsection{The paper could be written more conscisely. In particular the figures in the evaluation must be scaled up.}
            With the removal of elements linked to the TS policy, some room have been freed. This extra room will be used to scale up the figures.

        \subsubsection{Taking away FIQ from the guest OS further impacts performance. Please explain why you did not use Monitor mode, in particular since you do not need to spill many registers for executing}
            This matter is discussed in the fourth paragraph of the discussion section (section 7).
            The authors are not sure what the reviewer referres to by "Monitor mode", however, we beleive that the proposed FIQ routine is the fastest approach for this specific platform. On other platforms running different Operating System, mileage may vary.

        \subsubsection{Please explain why FP offers 16 priorities for 4 queues.}
            Short explanation added line 424-425. Typically, \schim offers up to 16 level of priority for this instance because the platform in used can support up to 16 colors and, by extension, up to 16 queues. Each of these queues require a dedicated level of priority.

    \subsection{Reviewer C}
        \subsubsection{The paper is excessively overselling the idea while underestimating the weaknesses.}


        \subsubsection{The paper does not solve in fact these problems, while introducing new ones. It does not solve these problems because still SchIM is implemented in PL which is running on order-of-magnitude slower clock frequency than the PS. This is clear from the results where SchIM kills the memory performance by more than 75\% throughput degradation (Figure 4).}


        \subsubsection{This solution drastically affects throughput as I highlighted earlier. It does the same for memory latency since a request/access through this solution has to navigate through the PL and SchIM. This runs at the PL clock; suffering several cycles from the solution (this is not fully quantified in the paper, which I discuss in next point),  will translate into a much larger number of CPU (or DRAM) cycles since those run at higher speeds (GHz compared to few MHz of PL). Accordingly, I am plotting that individual memory request latencies also increase by ~100s cycles.}


        \subsubsection{Given that the paper also addresses performance isolation and guarantees as a use case considering real-time systems as a potential customer, bounding memory latencies is also an important factor. There has been a plethora of research papers in the community to bound (and tighten) these latencies. This solution will definitely increase these latencies and that effect is not quantified or even highlighted in the paper.}

            \paragraph{Those schedulers are implemented on top of (or on the way to) the actual memory controller existing schedulers (such as FR-FCFS). Accordingly, they do not cancel its effect.}

            \paragraph{the system designer still needs to apply the existing techniques from literature to bound the interference of these memory controller schedulers.}

            \paragraph{With regard to the PL-to-PS feedback: It is not clear to the reviewer how the effectiveness of this is different from the natural backpressure/backlogging mechanisms deployed by most existing COTS SoCs?}

            \paragraph{With regard to the PL-to-PS feedback: Compared to the aforementioned already implemented techniques, this PL-to-PS feedback has its own drawbacks:}
