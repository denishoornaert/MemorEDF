\section{Introduction}


PS-PL platforms are surging in popularity among manufacturers,
researchers and industry practitioners, with commercially available
ARM-based products offered by, most notably, Intel~\cite{stratix10}
and Xilinx~\cite{ultrascale+}. A pilot large-scale, high-performance
PS-PL system is the Enzian platform~\cite{enzian20} being rolled out
by ETH Zurich\footnote{Also see
  \url{http://enzian.systems/}}. Recently, a RISC-V-based solution has
been made available by Microsemi --- the PolarFire
SoC~\cite{icikle_kit}. \todo[inline]{RM: we might want to move this
  part to the intro instead.}

This paper makes the following contributions:

1) We demonstrate that without any modification to the SoC circuitry,
a configurable module could be interposed between the core and memory
controller to perform traffic shaping. The proposed module operates at
the granularity of transactions based on their interarrival time.

2) We perform transaction-level memory scheduling in configurable
hardware. Various scheduling policies implemented in the module and
could be configured at run-time.

3) A groundbreaking view on memory access scheduling, namely MemorEDF,
has been proposed. In this method, the memory is viewed as a single
resource .....

4) We provide a run-time profiling interface to log the
transaction-level behavior of an application. Recorded data exploited
to make and enforce efficient scheduling decisions for EDF and LLF
policies.

5) This work implement a sandbox framework to test scheduling
platforms on real hardware. Newly proposed scheduling techniques can
be evaluated against the actual device, without rewiring the platform.

6) We implement and evaluate a full-stack design that includes the
memory scheduler hardware module. Using this implementation, we
analyze several well-known scheduling algorithms, namely, Earliest
Deadline First (EDF), Least Laxity First (LLF), Time-division
multipleaccess (TDMA).



One of the key limiting factors for the adoption of modern multi-core
embedded platforms in safety-critical systems is complexity. As this
class of platforms increase in heterogeneity
