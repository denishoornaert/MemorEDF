\section{Conclusion}
%    \todo[inline]{DH: As stressed by @Renato, we must mention that SchIM is the building brick for profile-based Bus scheduling/regulation!}
%    \todo[inline]{DH: Mention that killing the cores transparently at the hardware level opens the door for multiple other regulation scheme that would be based on traffic inspection}
%    \todo[inline]{DH: Mention that we only have three policies, but the amount of interesting policies can quickly be expanded. For instance, by simply combining the provided TDMA and FP, one can implement a sort of work conserving TDMA. A round robin policy can also be implemented.}
%    \todo[inline]{DH: Utilisation can be extended to hardware accelertors}
%    \par{\bf A stepping stone for predictive memory scheduling}
    
    In the present article we introduced \schim, a memory transactions scheduler framework that can be integrated in commercially available platform featuring a tightly coupled Processing System and Programmable Logic. Through a set of experiments, we assessed the the framework capabilities and demonstrated the correct behaviour of the proposed scheduling policies: Fixed Priority, Time Division Multiple Access and Traffic Shaping. Finally, we showed using a benchmarking suite the capability of \schim to ensure the isolation of cores under heavy contention.
    
    The authors see \schim as a stepping stone for predictive memory scheduling. The proposed framework having been design with extensibility in mind, advanced integration in commercially available platform as well as advanced scheduling techniques are few of many possible future directions. Especially, \schim is seen by the authors as the first step toward profile-based memory access scheduling.
