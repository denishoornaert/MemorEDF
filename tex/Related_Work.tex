\section{Related Work}
    \todo[inline]{Ask Tomasz and Gero for their ECRTS article}
    \todo[inline]{We should look at this paper "BRU: Bandwidth Regulation Unit for Real-Time Multicore Processors"}
    Citation list:
    \begin{itemize}
            \item PREM
            \item MemGuard
            \item PLIM
            \item Three Phase Model
            \item FRED Framework from Sant'Anna
     \end{itemize}
   
As multiple processors share individual shared recourses like memory, arbitration can occur at multiple levels of the system. Previous work in this area has traditionally been classified into \textbf{hardware-based}  and \textbf{software-based} techniques or the combination of the two \cite{assayad2010scheduler},\cite{morton2004hardware}. 

Toward achieving better and more predictable performance, a large body of works has addressed the problem by advancing novel hardware redesigns \cite{mutlu2007memory}, \cite{mutlu2007stall} , \cite{mutlu2008parallelism}, \cite{nesbit2006fair}  \cite{cho2005scheduler}, \cite{usui2016dash},\cite{ferri2011soc},\cite{zagan2017cpu},\cite{zhou2016mitts}, \cite{rafique2007effective}, \cite{gupta2010hardware}, \cite{moscibroda2008distributed}, \cite{kuacharoen2003configurable} mostly at the controller level. Hardware scheduler typically accelerates system performance at the cost of grown hardware resources, inflexibility, and integration hassle. Earlier endeavor by Åkesson et al. \cite{akesson2007predator}, \cite{akesson2010predictable} and Paolieri et al. \cite{paolieri2009analyzable} attains timing predictability through dynamically scheduling precomputed sequences of SDRAM commands. MEDUSA \cite{valsan2015medusa} exercises a two-level scheduling algorithm at the DRAM controller to render strong timing predictability. Another related study is a work done by Tang et al. in \cite{tang2014hardware}, wherein authors prototyped a configurable Task-Queue-based scheduler that fits various scheduling algorithms FPGA.


A plethora of techniques have approached by proposing software-only resolutions to alleviate performance interference.
SOFTWARE: ATLAS? / TCM -> fairness / OS-level techniques 
\todo[inline]{SR: to be completed}

Nevertheless, recently, with the advent of the fresh class of commercially available SoCs, integrated with a programmable logic (PL), there is a new third class of regulation.
\todo[inline]{SR: to be completed}


What sets this work apart from the literature surveyed above is
(1) \schim applies to new and traditional COTS system without introducing any hardware modification. Hence, keeping the intact systems while (2) providing a safe sandbox to test feasibility and realization of desired scheduling policies by providing (3) comprehensive run-time configuration interface to select between desired supported policies plus an option of bypassing the scheduler entirely.