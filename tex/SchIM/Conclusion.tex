\section{Conclusion}
%    \todo[inline]{DH: As stressed by @Renato, we must mention that SchIM is the building brick for profile-based Bus scheduling/regulation!}
%    \todo[inline]{DH: Mention that killing the cores transparently at the hardware level opens the door for multiple other regulation scheme that would be based on traffic inspection}
%    \todo[inline]{DH: Mention that we only have three policies, but the amount of interesting policies can quickly be expanded. For instance, by simply combining the provided TDMA and FP, one can implement a sort of work conserving TDMA. A round robin policy can also be implemented.}
%    \todo[inline]{DH: Utilisation can be extended to hardware accelertors}
%    \par{\bf A stepping stone for predictive memory scheduling}


In the present article we introduced the \schim, a memory transactions
scheduler framework that can be integrated in commercially available
platforms featuring a tightly coupled processing system and
programmable logic. A full-system implementation in a commercially
available PS-PL platform has been detailed, which encompasses the
accompanying software stack and the platform-specific integration
steps have been detailed in as well as advanced scheduling techniques
are few of many possible future directions.

Through a set of experiments, we assessed the capabilities of the
framework and demonstrated the correct behaviour of the proposed
scheduling policies, namely Fixed Priority, Time Division Multiple
Access and Traffic Shaping. Finally, we showed using a suite of
real-world benchmarks that the \schim is capable of enforcing strong
temporal isolation in spite of heavy memory contention. The only exception being the proposed Traffic Shapping scheduler.

The authors see the proposed \schim as a stepping stone to propose,
test, and validate novel memory scheduling policies to be tested on
embedded platforms with realistic performance and complex
workload. For this reason, the \schim has been designed to be
open-source and with extensibility in mind. Especially, we strongly
envision that the \schim could represent a stepping-stone toward
profile-based memory traffic scheduling.
