\documentclass[10pt,conference]{IEEEtran}
\IEEEoverridecommandlockouts
% The preceding line is only needed to identify funding in the first footnote. If that is unneeded, please comment it out.
\usepackage{cite}
\usepackage{amsmath,amssymb,amsfonts}
%\usepackage{algorithmic}
\usepackage{graphicx}
\usepackage{textcomp}
\usepackage{xcolor}
\usepackage{tikz}
\usepackage{multirow}
\usetikzlibrary{arrows.meta}
\usepackage{subcaption}
\usepackage{todonotes}
\usepackage{multirow}
\usepackage{xspace}
\usepackage{hyperref}
\usepackage{url}
\usepackage{comment}
\usepackage{listings}

\def\BibTeX{{\rm B\kern-.05em{\sc i\kern-.025em b}\kern-.08em
    T\kern-.1667em\lower.7ex\hbox{E}\kern-.125emX}}
\begin{document}

\newcommand\schim{SchIM\xspace}
\newcommand\schimL{Scheduler In-the-Middle\xspace}
\newcommand\schiml{scheduler in-the-middle\xspace}
\newcommand\axiin[1]{$\texttt{HPM}_{#1}$\xspace}
\newcommand\axiout[1]{$\texttt{HPS}_{#1}$\xspace}
\newcommand\axiconf[1]{$\texttt{LPM}_{#1}$\xspace}

\newcommand{\fig}[1]{Fig.~\ref{#1}}

\newcommand*\circledfig[2]{Fig.~\ref{#1}\tikz[baseline=0pt]{\node[anchor=south west,red,shape=circle,draw,inner sep=1pt] (char) {\scriptsize#2};}}

\newcommand*\circled[1]{\tikz[baseline=0pt]{\node[anchor=south west,red,shape=circle,draw,inner sep=1pt] (char) {\scriptsize#1};}}

\title{
    Work in Progress: The Cortex-A53 Last-Level of Cache Conundrum
%    \thanks{Identify applicable funding agency here. If none, delete this.}
}

\author{
%    \IEEEauthorblockN{1\textsuperscript{st} Given Name Surname}
%    \IEEEauthorblockA{
%        \textit{dept. name of organization (of Aff.)} \\
%        \textit{name of organization (of Aff.)}\\
%        City, Country \\
%        email address or ORCID
%    }
%    \and
%    \IEEEauthorblockN{2\textsuperscript{nd} Given Name Surname}
%    \IEEEauthorblockA{
%        \textit{dept. name of organization (of Aff.)} \\
%        \textit{name of organization (of Aff.)}\\
%        City, Country \\
%        email address or ORCID
%    }
%    \and
%    \IEEEauthorblockN{3\textsuperscript{rd} Given Name Surname}
%    \IEEEauthorblockA{
%        \textit{dept. name of organization (of Aff.)} \\
%        \textit{name of organization (of Aff.)}\\
%        City, Country \\
%        email address or ORCID
%    }
    Authors omitted for review.
}

\maketitle

\begin{abstract}
    In modern real-time multicore systems, understanding and adequatly managing shared caches is important to ensure the temporal isolation of critical tasks.
    Recent research have identify and extensively study the sources of unpredictability imputable to shared caches, heavily promoting techniques such as cache partitioning and internal sub-component management.

    In this article, we highlight the existence of an enigmatic source of inter-core interferences. Experiments realised on a development board show that benchmarks (issued from the San-Diego Vision Benchmark Suite) can see their execution time multiplied by 10. The same experiment shows that for extreme cases the core cluster can be stalled indefinitely.
\end{abstract}

\begin{IEEEkeywords}
    Multi-processors Systems, Real-Time Systems, Non-blocking Caches
\end{IEEEkeywords}

\section{Introduction}
    In modern embedded Mutli-core Systems, caches have become an angular piece of hardware bridging the gap between the speed of the connected execution units and the main memory. With the growing demand for high-performance multi-core system on chips, shared caches have evolved to accomodate the many concurrent accesses to main memory. These caches are referred to as \emph{non-blocking}.\\

    Unfortunately, while non-blocking shared caches offer great average perfomance, their behaviour is opaque and unpredictable. Dealing with the cache behaviour is of the utmost importance for safety critical hard Real-Time systems where timing constraints must be respected. A great deal of research has been conducted on cache management for Real-Time applications on MPSoCs. The two main sources of unpredictability imputed to the last level of cache are (1) the inter-core cache line eviction and (2) the opaque management of internaly shared resources.

    The inter-core cache line eviction is a well studied source of unpredictability that arises when the memory accesses of two independent cores lead to the eviction of each others cache line in a destructive way. Such source of unpredictability can be addressed and mitigated by enforcing the \emph{spacial isolation} of the cores. Both software solutions (e.g. via cache coloring \cite{}) and hardware solutions (e.g. via lockdown per master \cite{Giovani_cahe_partitioning_survey}) are available and commonly used.

    Inter-core interferences caused by internal shared resources such as the \emph{Miss Status Holding Registers} (or MSHR) have been recently studied. In \cite{Heechul_taming_non_blocking_caches} \cite{Heechul_DDOS_attacks_on_shared_cache}, the authors have shown that these shared resources can introduce a consequent amount of interference, mutiply the exeuction time of the tasks running ont he victim core by a factor of 346 (TODO check this). Such conditions only occur when the attacker creates extrem contention in the MSHR unit, resulting in a head-of-the-line-blocking.\\

    In the present article, we show that on the ARM Cortex-A53 \cite{ARM-cortex-A53} a third source of inter-core interferences exists: the target memory response time. In addition, we demonstrate that, in contrary to what has been shown before, read transactions can also caused interferrences. More accurately, we show that if a target memory ackownledges the transaction, but waits to deliver the response in a timely manner, the execution time of tasks running on independent cores can be impacted by a factor of 10. Furthermore, we show that if this single read transaction is acknowledged by the target memory, but the latter never provides a response, the whole core cluster is frozen indefinitely. To the best of our knowledge, this is the first report demonstrating that a core cluster can be subject to interferences caused by a single isolated read transaction.\\

    The present article is organised as follows: TODO

\section{Background}
    \subsection{Non-blocking Caches}
        Caches in modern MPSoC are an angular component that efficiently bridges the gap between the speed of the excution units and  the main memory.
        However, as good as they are at proving great bandwidth, \emph{blocking caches} suffers tremendously from the cache-miss penalty as they prevent the execution units to run as long as the data is not received from the main memory.
        In order to hide this penalty and improve the cache performance, \cite{Kroft} is the first to propose a \emph{Miss-Handling-Architecture} (MHA).
        This type of cache referred to as \emph{Non-blocking} relies on the introduction of a set of new registers called \emph{Miss Status Holding Register} (MSHR), which are in charge of tracking the status of cache line miss.
        Each MSHR stores inportant information regarding the cache-miss such as the target address and the location of the cache line to refill.
        For each level of cache in a system, the amount of MSHRs denotes the amount of outstanding (i.e. simultaneous) transactions the cache can handle.
        This amount is known as the \emph{Memory-Level-Parallelism} (MLP).

        At run time, a non-blocking cache behaves as follows. When a cache-miss occur, the metadata of the cache-miss is stored in one available MSHR (\emph{primary} request). In case the same cache-miss already occured and one MSHR already holds the metadata, the two requests are merged (\emph{secondary} request). It is only once the cache line refill request as been served and placed in the proper cache line that the MSHR is made available. Provided that all the MSHRs are used at a given instant, the system stops until one of them becomes available.

    \subsection{Programmable Logic In the Middle (PLIM)}
        The \emph{Programmable Logic In the Middle} (PLIM) is a new paradigm introduce by \cite{PLIM20} that takes advantage of the newly available platforms associating a traditional \emph{Processing System} (or PL side) and a tightly integrated peice of Programmable Logic (or PL side).
        In a system using PLIM, the PL side is leveraged such that the latter is located in between the core cluster and the main memory in the data path.
        In other words, a PLIM module is a peice if custom logic located on the PL side that is capable of manipulating the transaction comming from the core cluster before forwarding them to the main memory.
        For instance, \cite{PLIM20} tackled important constraints imposed by the cache coloring technique by using a PLIM module (called \emph{bleacher}) that manipulating each incomming transaction address.
        The use of a PLIM module broadens considerably the control on the traffic and the range of possibilities as it becomes possibe to manipulate the memory traffic generated by the core cluster at the granularity of a transaction.

\section{Related work}
    \begin{itemize}
        \item \cite{Heechul_taming_non_blocking_caches} Tmaing non blocking cache blablabla
        \item \cite{Heechul_DDOS_attacks_on_shared_cache} DDOD: attacks on shared cahe blablabla
    \end{itemize}

\section{System Model}
    \label{sec:system_model}
    \begin{figure*}
        \centering
        \includegraphics[scale=0.56]{images/Evaluation_setup.pdf}
%        \caption{Schematic view of the considered setup with the partitioning of the core cluster (both cores and LLC) on the left, the different path taken by the transactions highlighted in cyan, orange and red, and the AXI-Resistor. on the right.}
        \caption{Schematic view of the system model with the partitioned core cluster, the different data paths and the AXI-Resistor.}
        \label{fig:system_schematic}
    \end{figure*}

%    As a model, we consider a PS-PL system where the processing system must feature at least two cores and a shared non-blocking LLC.
%    The PL side is programmed with a PLIM module called the AXI-Resistor as depicted in Figure \ref{fig:system_schematic}, which will act as a slow cacheable memory target (more details in Section \ref{subsec:axi-resistor}).
%    The platform resources are shared between two actors: a \emph{victim} and an \emph{attacker}.
%    In the present scenario, the victim executes a set of tasks addressing directly the main memory (i.e. the path highlighted in red in Figure \ref{fig:system_schematic}), whereas the attacker mainly targets the AXI-Resistor with read transactions.
%    Figure \ref{fig:system_schematic} offers a schematic representation of the platform and how it is used for the experiment.

    As a model, we consider a PS-PL system where the processing system must feature at least two cores and a shared non-blocking LLC.
    The PL side is programmed with a PLIM module called the AXI-Resistor as depicted in Figure \ref{fig:system_schematic}, which will act as a slow cacheable memory target (more details in Section \ref{subsec:axi-resistor}).
    The platform resources are shared between two actors: a \emph{victim} and an \emph{attacker}.
    In the present scenario, the victim executes a set of tasks addressing the main memory (i.e., the path highlighted in red in Figure \ref{fig:system_schematic}), whereas the attacker mainly targets the AXI-Resistor with read transactions.
    Figure \ref{fig:system_schematic} offers a schematic representation of the platform and its use for the experiment.

%    This Section presents the details of the different components and actors of the experiment.
%    Further details regarding the organization of the PS side are given in Section \ref{subsec:processing_system_organization}.
%    In section \ref{subsec:attacker_reading_memory_bomb}, a description of the attacker is given.
%    Finally, the implementation and the characteristics of the AXI-Resistor are discussed in Section \ref{subsec:axi-resistor}.

    \subsection{Processing System Organization}
        \label{subsec:processing_system_organization}
%        As previously mentioned, the PS side and especially the core cluster is shared by both the victim and the attacker.
%        We assume a partitioned system where each actor is assigned a given set of cores and a private partition of the LLC.
%        Consequently, each actor is independent, ensuring that the observed delays cannot be imputed to either a common software stack or inter-core cache line evictions.
%
%        On one hand, we define our victim actor as a set of trusted applications and controllers having to meet certain deadlines.
%        On the other hand, we define our attacker actor as a lightweight application in charge of emitting sequential read transactions toward the desired target.
%        As shown in Figure \ref{fig:system_schematic}, the attacker splits its private partition of the LLC in two.
%        The first one half allows the attacker to access the main memory, where its code is located (via the path highlighted in red in Figure \ref{fig:system_schematic}) and the second half is dedicated to the data read through the AXI-Resistor (the orange path in Figure \ref{fig:system_schematic}).
%        Isolating the two address spaces is important as it enables a direct control over the amount of transactions targeting the AXI-Resistor.
%
%        Globally, the actors are perfectly isolated.
%        The only exception being that the attacker also access the main memory to fetch its code.
%        Nonetheless, this should introduce little or no inter-core interference.

        As previously mentioned, the PS side is shared by both the victim and the attacker.
        We assume a partitioned system where each actor is assigned a given set of cores and a private partition of the LLC.
        Consequently, each actor is independent, ensuring that the observed delays cannot be imputed to either a common software stack or inter-core cache line evictions.

        On one hand, we define our victim actor as a set of trusted applications and controllers having to meet specific deadlines.
        On the other hand, we define our attacker actor as a lightweight application in charge of emitting sequential read transactions toward the desired target.
        As shown in Figure \ref{fig:system_schematic}, the attacker splits its private partition of the LLC in two.
        The first half allows the attacker to access the main memory, where its code is located (via the path highlighted in red in Figure \ref{fig:system_schematic}). The second half is dedicated to the data read through the AXI-Resistor (the orange path in Figure \ref{fig:system_schematic}).
        Isolating the two address spaces enables a precise control over the number of transactions targeting the AXI-Resistor.

        Globally, the actors are perfectly isolated.
        The only exception is that the attacker also accesses the main memory to fetch its code.
        Nonetheless, this should introduce little or no inter-core interference.

    \subsection{Attacker - Reading Memory Bomb}
        \label{subsec:attacker_reading_memory_bomb}
%        Even with the aforementioned precautions, the design of the attacker (i.e. the read memory bomb) must be thought carefully.
%        In fact, if not under control a read memory bomb will steadily fetch data, creating many cache-misses.
%        Following the non-blocking cache mechanism, these cache-misses will be inserted in one of the available MSHRs until all of them are used.
%        In this situation the non-blocking cache controller will stop the whole machinery, leading to the phenomenon reported by \cite{Heechul_DDOS_attacks_on_shared_cache}.
%        %
%        This effect can only be avoided by throttling down the attacker core.
%        We enforce this by following each read request by a \emph{Data Synchronization Barrier} (\texttt{DSB}).
%        This ensures that at each instant, there will not be more than one transaction targeting the AXI-Resistor and, by extension, it guarantees at most one MSHR is occupied by the attacker actor.

        Even with the aforementioned precautions, the design of the attacker (i.e., the read memory bomb) must be thought carefully.
        In fact, if not under control, a read memory bomb will steadily fetch data, creating many cache-misses.
        Following the non-blocking cache mechanism, these cache-misses will be inserted in one of the available MSHRs until all of them are used.
        In this situation, the non-blocking cache controller will stop the whole machinery, leading to the phenomenon reported by \cite{Heechul_DDOS_attacks_on_shared_cache}.

        This effect can only be avoided by throttling down the attacker core.
        We enforce this by following each read request by a \emph{Data Synchronization Barrier} (\texttt{DSB}).
        This instruction ensures that at each instant, there will not be more than one transaction targeting the AXI-Resistor and, by extension, it guarantees at most one MSHR is occupied by the attacker actor.

    \subsection{AXI-Resistor IP}
        \label{subsec:axi-resistor}
%        In our system model, the AXI-resistor IP is a PLIM module \cite{PLIM20} used to act as a slow cacheable memory target.
%        Typically, the IP accepts every read transaction coming from the core cluster via the HPM port, buffers them and only release them one by one in direction of the DRAM controller.
%        Releases are spaced by a minimal inter-arrival time (MIT) expressed in clock cycles (CC).
%        This data path is highlighted in orange in Figure \ref{fig:system_schematic}.
%        Because each transaction is intercepted by the AXI-Resistor before arriving to the DRAM controller, the latter is unaware of the transaction and no internal mechanism is activated, suppressing potential interference induced by the DRAM controller.
%
%        The AXI-Resistor is composed of three ports: one slave port accepting transactions from the core cluster, one configuration port (blue route in Figure \ref{fig:system_schematic}) and one master port relaying the transactions out of the IP.
%        Arriving transactions are buffered within the AXI-Resistor thanks to a queue (see \emph{Addresss Phase Lane} on the right of Figure \ref{fig:system_schematic}).
%        The transaction stored at the head of the queue is released according a timer.
%        The period of this timer is reprogrammable at run-time thanks to the configuration port.
%        Once a read request has been served by the DRAM controller, the read data is sent back to the core cluster through the AXI-Resistor.
%        This phase is not buffered by the IP as shown in Figure \ref{fig:system_schematic} (see \emph{Response Phase Lane}).

        In our system model, the AXI-resistor IP is a PLIM module \cite{PLIM20} used to act as a slow cacheable memory target.
        Typically, the IP accepts every read transaction coming from the core cluster via the HPM port, buffers them, and only releases them one by one toward the DRAM controller.
        Releases are spaced by a minimal inter-arrival time (MIT) expressed in clock cycles (CC).
        This data path is highlighted in orange in Figure \ref{fig:system_schematic}.
        Because the AXI-Resistor intercepts each transaction before they reach the DRAM controller, the latter is unaware of the transactions. Consequently, no internal mechanism is activated, suppressing potential interference induced by the DRAM controller.

        The AXI-Resistor is composed of three ports: one slave port accepting transactions from the core cluster, one configuration port (blue route in Figure \ref{fig:system_schematic}), and one master port relaying the transactions out of the IP.
        Arriving transactions are buffered within the AXI-Resistor thanks to a queue (see \emph{Addresss Phase Lane} on the right of Figure \ref{fig:system_schematic}).
        The transaction stored at the head of the queue is released according to a timer.
        The period of this timer is reprogrammable at run-time thanks to the configuration port.
        Once the DRAM controller has served a read request, the data read is sent back to the core cluster through the AXI-Resistor.
        This phase is not buffered by the IP as shown in Figure \ref{fig:system_schematic} (see \emph{Response Phase Lane}).

\section{Evaluation}
    \label{sec:evaluation}
    For our experiment, we use the Xilinx ZCU102 develpment board \cite{Xilinx-ULTRASCALE-TRM}, a PS-PL platform featuring four ARM Cortex-A53 cores \cite{ARM-cortex-A53} connected together thanks to a shared LLC of 1MB.
    Thanks to the Jailhouse hypervisor, we are able to isolate the actors by assigning them specific cores and private partitions of the LLC.
    The victim inmate is allocated three cores and half of the LLC.
    It runs a selection of micro benchmarks issued from the San-Diego Vision Benchmark Suite \cite{SD-VBS} on top of Linux.
    The attacker is a lightweight baremetal inmate running on the remaining core.
    Finally, the PL side (and subsequently the AXI-Resistor) is clocked at 250MHz.

    \begin{figure*}
        \centering
        \includegraphics[scale=0.425]{images/cpu-brainfreeze-interference.pdf}
        \caption{Normailized execution time for different combinations of benchmark (inset), input sizes (x axis) and MITs (bar).}
        \label{fig:cpu-brainfreeze-interference-results}
    \end{figure*}

    Using the scenario and setup presented in section \ref{sec:system_model}, we aim at observing the interference caused bu the lightweight read attcker on the victim inmate. To this end, we ran few applicative benchmarks issued from the San-Diego Vision Benchmark Suite \cite{SD-VBS} for all the available input sizes. The benchmarks were also run ffor different configuration of the AXI-Resistor. The latter was configured with MITs of $2^{10}$, $2^{15}$, $2^{17}$ and $2^{19}$. As a baseline, the benchmarks have also been run alone (i.e. without an attacker). This baseline is referred to as "Solo" in Figure \ref{fig:cpu-brainfreeze-interference-results}.

    Figure \ref{fig:cpu-brainfreeze-interference-results} offers a relevant subset of results obtained from the experiment. All the results for a given benchmark (localization, mser, sift and tracking) and its size input have been normalized with respect to the equivalent combination running alone (i.e. Solo, the leftmost blue bar in each bar cluster).

    As display in Figure \ref{fig:cpu-brainfreeze-interference-results}, the victim inmate, while running the localization benchmark, suffers from little to no interference from the attacker. The only noticeable increase in execution time is for a \emph{vga} input size and a MIT of $2^{19}$, where the victim suffers from a $150\%$ increase of computation time.
    On the other hand, when running mser, a well known memory bound benchmark, noticeable increases in execution time are observed, with a factor of 10 observed for the \emph{sim} input size.
    For both sift and tracking spikes in execution time are observed. Eventhough, there is no clear pattern with respect to the input sizes, increase in execution time happen for MITs of $2^{17}$ and $2^{19}$.\\

    Interrestingly, configuring the AXI-Resistor in such a way that it would accept the read trasaction, but never answer systematically, leads to the whole system being suspended undefinitely. This can effectively be considered as a Denial-of-Service attack.

\section{Discussion}
    Non-blocking caches are advertised as cache units capable of hiding the cache hit-miss penalty and managing multiple simultaneous memory accessses created by the cores in a seamless fashion (i.e. without stalling the whole core cluster at each miss) unless all the MSHRs are used or the write-back unit buffer is full. Nothing in the non-blocking cache architecture suggests that a single outstanding read transaction could introduce inter-core interferences. However, our experiment tends to show the opposite.

    While the exact source of the observed inter-core interferences is unclear to the authors, all the precautions taken during the experiment (i.e. isolation of the inmates and partition of the cache) and the result tend to suggest that the source originates from the LLC controller itself.

    The authors acknowledge that the described phenomenon is unlikely to occur in a normal situation (i.e. all the inmates target the main memory), and if it does, the consequences should be negligeable.
    Nonetheless, this experiment has the merit of pinpointing a malfunction in the LLC controller of the ARM Cortex-A53.
    It is a reminder of the gap between the theoretical models, the hardware behaviour expectations and the real behaviour of the hardware.

%    The experiment also sheds light on the importance of selecting trustable third party IPs and designing correctly IPs if they aim to be cacheable target memories.
%    Any bus slaves must be designed carefully in order to provide fast answers. This is specially the case for PLIM modules which aim at being cacheable targets.
%    Finally, software stacks provided with SoCs featuring a tightly integrated programmable logic must ensure that the latter can only be reprogrammed by a trusted actor as simply holding a single cached read transaction can indefinitely stall the whole core cluster.

\section{Conclusion}
    \begin{itemize}
        \item The authors acknowledge that the described phenomenon is unlikely to happen in a mainstream platform, or if it happens, the consequences are negligeable. Nonethless, this experiment has the merit of pinpointing a malfunction in the last level of cache controller of the ARM Cortex-A53. The experiment also shed light on the importance of selecting trustable 3thr party IPs and designing correctly IPs if they are cacheable target memories.
        \item Such system, under strict conditions cannot quarantee QoS nor Mixed-criticality levels.
        \item In contrast to what has been previously reported, read intensive applications can also become a threat for the system predictability.
        \item Bus slaves must be designed carefully to provide fast answers. More specifically, SoCs featuring a tightly integrated FPGA must ensure that the FPGA can only be reprogrammed by a trusted actor as simply holding a single transaction can indefinitely stall the whole core cluster.
    \end{itemize}

    \subsection{Future works}
        \begin{itemize}
            \item Study if the same impact can be found with the attacker targeting a scratch pad memory which frequency could be modify.
        \end{itemize}


\bibliographystyle{IEEEtranS}
\bibliography{references}

\end{document}
