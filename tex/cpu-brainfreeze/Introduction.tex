\section{Introduction}
    \begin{itemize}
        \item In modern Heterogenuous Mutli-core Systems, caches are an angular piece of hardware as they efficiently bridge the gap between the speed of the connected execution units and the main memory.
        \item With the time, shared last-level of caches architecture has evolved to the point they are now capable of managing transactions to the main memory transparently. A technic known as non-blocking.
        \item While providing top of the line performances, this complex machinery is unpredictable, a source of concern for safety critical hard real-time systems.
        \item In addition to well known sources of unpredictability such as the inter-core eviction, recent research have highlighted that internal component such as the Miss-Status-Holding-Register (or MSHR) can introduce substantial inter-core interferences in certain circumstances.
        \item Previous research put in evidence that such situation occur solely under high write loads.
        \item The present article shows that the under certain circumstances, a single read transaction is also capable of jeoparding the system predictability and even fully block all the masters connected to the last-level of cache.
        \item We advocate that, in addition to the inter-core eviction and the management of shared LLC sub-units, a third source of unpredictability exists: the memory target response time.
    \end{itemize}
