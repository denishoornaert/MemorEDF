\section{Introduction}
%    In modern high-performance mutliprocessor system-on-a-chips (MPSoCs), caches have become a crucial piece of hardware bridging the gap between the speed of the processing elements and the main memory.
%    With the growing demand for high-performance systems, shared caches have evolved to accommodate for several concurrent accesses to main memory and hide the cache-miss penalty.
%    These caches are referred to as \emph{non-blocking}.
%
%    Unfortunately, while non-blocking shared caches offer high average bandwidth, their behaviour is opaque and unpredictable.
%    Understanding the cache behaviour is of the utmost importance for safety critical hard real-time systems where timing constraints must be respected and guaranteed.
%    %For instance, the \emph{Federal Aviation Administration} (FAA) \cite{faa} still recommends the use of single core platform until a sound analysis of the interference channels has been conducted.
%    For instance, the \emph{Federal Aviation Administration} (FAA) mandates the use of a single processor unless the impact of all the temporal interference channels existing in multi-core platforms can be appropriately identified, mitigated or bounded.
%
%    A great deal of research has been conducted on cache management for real-time applications on MPSoCs.
%    The two main sources of unpredictability imputed to the \emph{last-level cache} (LLC) are (1) the inter-core cache line eviction and (2) the opaque management of internally shared resources.
%    %
%    The inter-core cache line eviction is a well studied source of unpredictability that arises when the memory accesses of two independent cores lead to the eviction of each other cache lines in a destructive way.
%    Such source of unpredictability can be prevented by enforcing the \emph{spatial isolation} of the cores through \emph{way-based} or \emph{set-based} partitioning \cite{Mancuso2013RealtimeCM, 6755286, Giovani_cahe_partitioning_survey}.
%    %
%    Inter-core interferences caused by internal shared resources such as the \emph{Miss-Status-Holding-Registers} or the \emph{write-back} unit have been recently studied in \cite{Valsan2017AddressingIC, Heechul_DDOS_attacks_on_shared_cache}.
%    If left unmanaged, the contention on these resources can create important interferences even if the cores are spatially isolated.\\

%    In this article, we show the existence of a third source of inter-core interference linked to the speed at which the cacheable target memory reacts.% on the ARM Cortex-A53 \cite{ARM-cortex-A53}.
%    In other words, if a target memory acknowledges the transaction, but waits to deliver the response, the execution time of tasks running on independent cores (co-runners) can be impacted.
%    Our experiments show that a co-running task can see its execution being increased by a factor of 10.
%    Furthermore, we show that if a single read transaction is acknowledged by the target memory, but the latter never provides a response, the whole core cluster is frozen indefinitely.
%    To the best of our knowledge, this is the first report demonstrating that a core cluster can be subject to interferences caused by an isolated and delayed read transaction.

    In modern high-performance mutliprocessor system-on-a-chips (MPSoCs), caches have become a crucial piece of hardware bridging the gap between the speed of the processing elements and the main memory.
    With the growing demand for high-performance systems, shared caches have evolved to accommodate several concurrent accesses to main memory and hide the cache-miss penalty.
    These caches are referred to as \emph{non-blocking}.

    Unfortunately, while non-blocking shared caches offer high average bandwidth, their behavior is opaque and unpredictable.
    Understanding the cache behavior is of the utmost importance for safety-critical hard real-time systems where timing constraints must be respected and guaranteed.
    %For instance, the \emph{Federal Aviation Administration} (FAA) \cite{faa} still recommends the use of single core platforms until a sound analysis of the interference channels has been conducted.
    For instance, the \emph{Federal Aviation Administration} (FAA) mandates the use of a single processor unless the impact of all the temporal interference channels existing in multi-core platforms can be appropriately identified, mitigated, or bounded.

    A great deal of research has been conducted on cache management for real-time applications on MPSoCs.
    The two main sources of unpredictability imputed to the \emph{last-level cache} (LLC) are (1) the inter-core cache line eviction and (2) the opaque management of internally shared resources.

    The inter-core cache line eviction is a well-studied source of unpredictability that arises when the memory accesses of two independent cores lead to the eviction of each other cache lines in a destructive way.
    Such source of unpredictability can be prevented by enforcing the \emph{spatial isolation} of the cores through \emph{way-based} or \emph{set-based} partitioning \cite{Mancuso2013RealtimeCM, 6755286, Giovani_cahe_partitioning_survey}.

    Inter-core interferences caused by internal shared resources such as the \emph{Miss-Status-Holding-Registers} or the \emph{write-back} unit have been recently studied in \cite{Valsan2017AddressingIC, Heechul_DDOS_attacks_on_shared_cache}.
    If left unmanaged, the contention on these resources can create important interferences even if the cores are spatially isolated.\\

    In this article, we show the existence of a third source of inter-core interference linked to the speed at which the cacheable target memory reacts.
    In other words, if a target memory acknowledges a transaction but waits to deliver the response, the execution time of tasks running on independent cores (co-runners) is impacted.
    Our experiments show that a co-running task can see its execution increased by a factor of 10.
    Furthermore, we show that if the target memory acknowledges a single read transaction but never provides a response, the whole core cluster is frozen indefinitely.
    To the best of our knowledge, this is the first report demonstrating that a core cluster is subject to interferences caused by an isolated and delayed read transaction.
