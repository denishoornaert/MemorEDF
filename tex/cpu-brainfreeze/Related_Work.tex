\section{Related work}
    An important amount of research hass focused on addressing the challenges of isolating the the execution time of cores sharing the same cache.
    Most of this research \cite{Mancuso2013RealtimeCM, 6755286} has aimed at \emph{spacially isolate} the cores (i.e. avoiding inter-core cache line eviction by constraining each core dta and instruction in a specific region of the shared cache). Hardware based solutions such as \emph{lockdown per way} \cite{Giovani_cahe_partitioning_survey} are efficient, but not integrated in every platforms.
    On the other hand, software based solutions such as cache coloring \cite{Mancuso2013RealtimeCM} can be deployed in most plattforms, but comes at the cost of increased memory space requirements.

    However, recent research \cite{Valsan2017AddressingIC, Heechul_DDOS_attacks_on_shared_cache} have highlighted that, while cache partitioning is successful in most cases, in extreme cases, contention on shared internal units such as the MSHRs or the write-back unit can also introduce substential inter-core interferences.
    In \cite{Valsan2017AddressingIC}, the authors evaluate the impact of intercore interferrence originated by the MSHRs on multiple platforms and propose a solution to eliminate this contention. The solution is based on a combination of a small hardware module and an OS-level controller.
    Their experiments show that, if let unmanaged, the execution time of independent cores is increased respectivelly by a factor of 10.6 and 21.3 under read and write workloads.
    By the mean of simulation, they prove that their approach is successful at providing the best overall throughput for each core while suppressing the inter-core interference caused by the MSHRs.
    % Running applicative benchmarks, increas in execution time of a factor of 6.4 have been observed.
    \cite{Heechul_DDOS_attacks_on_shared_cache} investigates the contention in caches caused by shared internal units in the case of \emph{Denial-of-service} (DOS) attacks and propose an OS-level solution enabling finer management of the system bandwidth.
    In contrast to \cite{Valsan2017AddressingIC}, the internal unit studied and exploited is the write-back unit.
    They report that, by exploiting this unit efficiently, one can increase the execution time of a victim task by a factor of 346.
