\section{Related work}
%    A sizable amount of research has focused on addressing the challenges of isolating the cores sharing the same cache and thus preventing unpredictable temporal behaviour.
%    Most of this research \cite{Mancuso2013RealtimeCM, 6755286} has aimed at \emph{spacially isolating} the cores (i.e. avoiding inter-core cache line eviction by constraining each core data and instructions in a specific region of the shared cache).
%    Hardware based solutions such as \emph{lockdown per way} \cite{Giovani_cahe_partitioning_survey} are efficient, but not integrated in every platform.
%    On the other hand, software based solutions such as cache coloring \cite{Mancuso2013RealtimeCM, determ_virt} can be deployed on most platforms, but come at the cost of increased memory space requirements.
%
%    However, recent research \cite{Valsan2017AddressingIC, Heechul_DDOS_attacks_on_shared_cache} has highlighted that, while cache partitioning is successful in most cases, in some situations contention on shared internal units such as the MSHRs or the write-back unit can also introduce substantial inter-core interferences.
%    In \cite{Valsan2017AddressingIC}, the authors evaluate the impact of inter-core interference originated at the MSHRs on multiple platforms and propose a solution to eliminate this contention. The solution is based on a combination of a small hardware module and an OS-level controller.
%    Their experiments show that, if left unmanaged, the execution time of independent cores is multiplied by 10.6 and 21.3 under read and write workloads, respectively.
%    Via simulation, they prove that their approach is successful at providing the best overall throughput for each core while mitigating the inter-core interference caused by the MSHRs.
%    % Running applicative benchmarks, increas in execution time of a factor of 6.4 have been observed.
%    \cite{Heechul_DDOS_attacks_on_shared_cache} investigates the contention in caches caused by shared internal units in the case of \emph{Denial-of-service} (DOS) attacks and propose an OS-level solution enabling finer management of the system bandwidth.
%    In contrast to \cite{Valsan2017AddressingIC}, the internal unit studied and exploited is the write-back unit.
%    They report that, by exploiting this unit efficiently, one can increase the execution time of a victim task by a factor of 346.
%
%    Unlike the interference discussed in \cite{Valsan2017AddressingIC, Heechul_DDOS_attacks_on_shared_cache}, the CPU-brainfreeze interference does not consist in saturating shared registers. Instead, the interference reported in this article arises when a unique outstanding transaction is left unserved by the target memory for a long period of time.

    A sizable amount of research has focused on addressing the challenges of isolating the cores sharing the same cache in order to prevent unpredictable temporal behavior.
    Most of this research \cite{Mancuso2013RealtimeCM, 6755286} has aimed at \emph{spacially isolating} the cores (i.e., avoiding inter-core cache line eviction by constraining each core data and instructions in a specific region of the shared cache).
    Hardware-based solutions such as \emph{lockdown per way} \cite{Giovani_cahe_partitioning_survey} are efficient, but not integrated in every platform.
    On the other hand, software-based solutions such as cache coloring \cite{Mancuso2013RealtimeCM, determ_virt} can be deployed on most platforms, but come at the cost of increased memory space requirements.

    However, recent research \cite{Valsan2017AddressingIC, Heechul_DDOS_attacks_on_shared_cache} has highlighted that, while cache partitioning is successful in most cases, in some situations, contention on shared internal units such as the MSHRs or the write-back unit can also introduce substantial inter-core interferences.
    In \cite{Valsan2017AddressingIC}, the authors evaluate the impact of inter-core interference originated at the MSHRs on multiple platforms and propose a solution to eliminate this contention. The solution is based on a combination of a small hardware module and an OS-level controller.
    Their experiments show that, if left unmanaged, the execution time of independent cores is multiplied by 10.6 and 21.3 under read and write workloads, respectively.
    Via simulation, they prove that their approach is successful at providing the best overall throughput for each core while mitigating the inter-core interference caused by the MSHRs.
    \cite{Heechul_DDOS_attacks_on_shared_cache} investigates the contention in caches caused by shared internal units in the case of \emph{Denial-of-service} (DOS) attacks and propose an OS-level solution enabling finer management of the system bandwidth.
    In contrast to \cite{Valsan2017AddressingIC}, the internal unit studied and exploited is the write-back unit.
    They report that, by exploiting this unit efficiently, one can increase the execution time of a victim task by a factor of 346.
